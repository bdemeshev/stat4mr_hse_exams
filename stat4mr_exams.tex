% arara: xelatex: {shell: yes}
%% arara: biber
%% arara: xelatex: {shell: yes}
%% arara: xelatex: {shell: yes}


\documentclass[11pt, a4paper]{article}

\usepackage{fontspec}

\usepackage[base]{babel} % hypenation
% see: https://tex.stackexchange.com/questions/400986/hyphenrules-environment-no-longer-works-with-polyglossia
\usepackage{polyglossia}

% \setdefaultlanguage{russian}
\setmainlanguage{english}
\setotherlanguages{russian}

% download "Linux Libertine" OTF-fonts:
% http://www.linuxlibertine.org/index.php?id=91&L=1
\setmainfont{Linux Libertine O} % or Helvetica, Arial, Cambria
% why do we need \newfontfamily:
% http://tex.stackexchange.com/questions/91507/
\newfontfamily{\cyrillicfonttt}{Linux Libertine O}
\newfontfamily{\cyrillicfont}{Linux Libertine O}
\newfontfamily{\cyrillicfontsf}{Linux Libertine O}
 
\usepackage{etoolbox} % to use ifdef, must be after babel

\newtoggle{excerpt}
\togglefalse{excerpt}  % помечаем, что это не отрывок, а далее в тексте может использовать
 

\usepackage[paper=a4paper,
top=15mm,
bottom=13.5mm,
left=13mm, right=13mm, includefoot]{geometry}

\usepackage{etex} % расширение классического tex
% в частности позволяет подгружать гораздо больше пакетов, чем мы и займёмся далее




\usepackage{makeidx} % для создания предметных указателей
\usepackage{verbatim} % для многострочных комментариев
%\usepackage[pdftex]{graphicx} % для вставки графики
% omit pdftex option if not using pdflatex

\usepackage{comment} % для команды excludecomment


%\usepackage{dsfont} % шрифт для единички с двойной палочкой (для индикатора события)
\usepackage{bbm} % шрифт - двойные буквы


\usepackage[usenames, dvipsnames, svgnames, table, rgb]{xcolor}

\usepackage{colortbl}


% пакет для тестов:
% \usepackage[box, % запрет на перенос вопросов
% nopage, % убираем колонтитулы страницы
% insidebox, % ставим буквы в квадратики
% separateanswersheet, % добавляем бланк ответов
% nowatermark, % отсутствие надписи "Черновик"
% indivanswers,  % показываем верные ответы
% answers,
% lang=RU, % локализация слов "нет верных ответов", "вопрос" и тд
% completemulti % добавлять "нет правильного ответа" во всех вопросах множественного выбора
% ]{automultiplechoice}


\usepackage[colorlinks, hyperindex, unicode, breaklinks]{hyperref} % гиперссылки в pdf

\usepackage{amssymb}
\usepackage{amsmath}
\usepackage{amsthm}
\usepackage{epsfig}
\usepackage{bm}
\usepackage{color}

\usepackage{multicol}
\usepackage{multirow} % Слияние строк в таблице

\usepackage{textcomp}  % Чтобы в формулах можно было русские буквы писать через \text{}

%\usepackage{embedfile} % отказались от внедрения тех внутрь pdf так как всё равно всё на гитхабе :)

\usepackage{physics} % \abs \norm \grad, меняет \sin, \cos...

\usepackage{subfigure} % для создания нескольких рисунков внутри одного

\usepackage{tikz, pgfplots} % язык для рисования графики из latex'a
\usetikzlibrary{trees} % прибамбас в нем для рисовки деревьев
\usetikzlibrary{arrows} % прибамбас в нем для рисовки стрелочек подлиннее
\usetikzlibrary{automata, arrows, positioning, calc}
\usepackage{tikz-qtree} % прибамбас в нем для рисовки деревьев




\usepackage{enumitem} % развернутые списки

% свешиваем пунктуацию (т.е. знаки пунктуации могут вылезать за правую границу текста, при этом текст выглядит ровнее)
\usepackage{microtype}

% более красивые таблицы
\usepackage{booktabs}
% заповеди из докупентации:
% 1. Не используйте вертикальные линни
% 2. Не используйте двойные линии
% 3. Единицы измерения - в шапку таблицы
% 4. Не сокращайте .1 вместо 0.1
% 5. Повторяющееся значение повторяйте, а не говорите "то же"

\usepackage[cache=false]{minted} % вставка кода, нужен питон :)
% опция cache=false включена, чтобы избегать необходимости
% чистить кэш при ошибках компиляции
% возможно вообще подумать об устранении этого пакета:
% вставок кода мало, а эта зависимость (pygmentize + python)
% резко затрудняет редактирование новичкам
% может обойтись listings?

\usepackage{epigraph}

\usepackage{titleps} % заголовки страниц




% по поводу заголовков разделов в колонтитулах
% https://tex.stackexchange.com/questions/236705
% поэтому выбрали titleps вместо fancyhdr

\newpagestyle{mypage}{%
  \headrule
  \sethead{\sectiontitle}{}{\subsectiontitle}
  \setfoot{}{}{\thepage}
}
\settitlemarks{section,subsection,subsubsection} % !!!!!!no space after comma!!!!!!
\pagestyle{mypage}





\DeclareMathOperator{\Lin}{\mathrm{Lin}}
\DeclareMathOperator{\Linp}{\Lin^{\perp}}
\DeclareMathOperator*\plim{plim}
\DeclareMathOperator{\card}{card}
\DeclareMathOperator{\sgn}{sign}
\DeclareMathOperator{\sign}{sign}

\DeclareMathOperator*{\argmin}{arg\,min}
\DeclareMathOperator*{\argmax}{arg\,max}
\DeclareMathOperator*{\amn}{arg\,min}
\DeclareMathOperator*{\amx}{arg\,max}
\DeclareMathOperator{\cov}{Cov}
\DeclareMathOperator{\Var}{Var}
\DeclareMathOperator{\Cov}{Cov}
\DeclareMathOperator{\Corr}{Corr}
\DeclareMathOperator{\pCorr}{pCorr}
\DeclareMathOperator{\E}{\mathbb{E}}
\let\P\relax
\DeclareMathOperator{\P}{\mathbb{P}}


\newcommand{\cN}{\mathrm{N}}
\newcommand{\cU}{\mathrm{U}}
\newcommand{\cBinom}{\mathrm{Binom}}
\newcommand{\cBin}{\cBinom}
\newcommand{\cExp}{\mathrm{Exp}}
\newcommand{\cPois}{\mathrm{Pois}}
\newcommand{\cBeta}{\mathrm{Beta}}
\newcommand{\cGamma}{\mathrm{Gamma}}

\newcommand \R{\mathbb{R}}
\newcommand \N{\mathbb{N}}
\newcommand \Z{\mathbb{Z}}





\newcommand{\dx}[1]{\,\mathrm{d}#1} % для интеграла: маленький отступ и прямая d
\newcommand{\ind}[1]{\mathbbm{1}_{\{#1\}}} % Индикатор события
%\renewcommand{\to}{\rightarrow}
\newcommand{\eqdef}{\mathrel{\stackrel{\rm def}=}}
\newcommand{\iid}{\mathrel{\stackrel{\rm i.\,i.\,d.}\sim}}
\newcommand{\const}{\mathrm{const}}


% вместо горизонтальной делаем косую черточку в нестрогих неравенствах
\renewcommand{\le}{\leqslant}
\renewcommand{\ge}{\geqslant}
\renewcommand{\leq}{\leqslant}
\renewcommand{\geq}{\geqslant}



\AddEnumerateCounter{\asbuk}{\russian@alph}{щ} % для списков с русскими буквами
% \setlist[enumerate, 2]{label=\asbuk*),ref=\asbuk*} % \asbuk* \alph* \arabic*




% делаем короче интервал в списках
\setlength{\itemsep}{0pt}
\setlength{\parskip}{0pt}
\setlength{\parsep}{0pt}

% \newenvironment{problem}{}{}
% тут перещёлкиваем комментарий, чтобы убрать или показать решения:
% \newenvironment{sol}{}{} % with solutions
% \excludecomment{sol} % without solutions



\unitlength=0.6mm

\title{Statistics for Market Research exams}
\date{\today}
\author{Angry Teachers, Folklore}


%%%%%%%%%%%%%%%%%% вставки
%%%%%%%%%%%%%%%%%%%%%%%%%%%%%%%%%%%%%%% Списки без уродских отступов
\newenvironment{enumerate*}{
\begin{enumerate}
  \setlength{\itemsep}{0pt}
  \setlength{\parskip}{0pt}
  \setlength{\parsep}{0pt}
}{\end{enumerate}}

\newenvironment{itemize*}{
\begin{itemize}
  \setlength{\itemsep}{0pt}
  \setlength{\parskip}{0pt}
  \setlength{\parsep}{0pt}
}{\end{itemize}}

\abovedisplayskip=0mm
\abovedisplayshortskip=0mm
\belowdisplayskip=0mm
\belowdisplayshortskip=0mm


% https://tex.stackexchange.com/questions/241343
% https://tex.stackexchange.com/questions/168480
\emergencystretch 5em % разрешаем дополнительные пробелы для упаковки параграфа до правой границы


% \setcounter{secnumdepth}{0} % убираем нумерацию секций, подсекций и т.д.


%%%%%%%%%%%
% блок для тестов
%%%%%%%%%%%
% [1][3] 1 = one argument, 3 = value if missing
% эта магия создаёт окружение answerlist
% именно в окружении answerlist записаны варианты ответов в подключаемых exerciseXX
% просто \begin{answerlist} сделает ответы в три столбца
% если ответы длинные, то надо в них руками сделать
% \begin{answerlist}[1] чтобы они шли в один столбец
\newenvironment{answerlist}[1][3]{
\begin{multicols}{#1}
\begin{enumerate}[label=\fbox{\emph{\Alph*}},ref=\emph{\alph*}]
}
{
\end{enumerate}
\end{multicols}
}


\excludecomment{solution} % without solutions

\theoremstyle{definition}

% опция [subsection] для сброса счётчика вопросов после каждой subsection
\newtheorem{question}{Вопрос}[subsection]

% чтобы номер вопроса был без номера секции:
\renewcommand{\thequestion}{\arabic{question}}
% конец блока для тестов
%%%%%%%%%%%%









\begin{document}
\maketitle

%\clearpage
%\thispagestyle{empty}
\tableofcontents{}


\parindent=0 pt % no indent

\clearpage
\section*{Description}

See updates at \url{https://github.com/bdemeshev/stat4mr_exams}.

Click on red hyperlinks inside pdf, you can get to the answers and back!


Any comments? Bugs?
\url{https://github.com/bdemeshev/stat4mr_exams/issues/}.

The order of topics has changed after the first course iteration in 2020-21.
The interested reader may find relevant exercises by looking through all 2020-21 exams. 


\subsection*{Greatings to the contributors}

Here we describe only the style guidelines and typical erros. 
For more information on tex one may read the 
\href{http://www.ccas.ru/voron/download/voron05latex.pdf}{book} by K. Vorontsov.

\begin{enumerate}

\item Use decimal point as a separator: $3.14$ — good style, $3{,}14$ — bad style.
This goes against russian tradition, but favors copy-pasting numbers in software for computations. 
% \item Существует длинное тире, —, которое отличается от просто дефиса - и нужно,
% чтобы разделять части предложения, \href{https://ru.wikihow.com/напечатать-тире}{Инструкция
% в картинках по набору тире :)}
\item Use \verb|\[|\ldots\verb|\]| for display math formulas. Do not use \$\$\ldots\$\$!
\item Use \verb|cases| for systems of equations,
\verb|align*| for multiline formulas, \verb|enumerate| for enumerations.
\item Inside formulas use \verb|\text{|\ldots\} to write text.
\item Use \verb|\ldots| for ellipsis.
\item You can find useful macros in the preamble, like \verb|\P, \E, \Var, \Cov, \Corr, \cN|.
\item Use backslash before functions: \verb|\ln, \exp, \cos|\ldots
\item Use booktabs style for tables. You may use online \href{https://www.tablesgenerator.com}{tablesgenerator}. 
Choose booktabs table style instead of default table style.
\item Respect the letter ё! :)
\item Start every sentence in tex source from a new line. 
There will be no additional newlines in final pdf but tex file will be easier to read.
% \item В перечислениях после «Найдите» используй в качестве знаков препинания точки
% с запятой и точку в конце.
\item For multiplication use \verb|\cdot|. Please never use \verb|*| :)
\end{enumerate}

% стандарт имени файла:
% добавляется _sol в файле с решениями

%\input{chapters/october.tex}
%\input{chapters/december.tex}
%\input{chapters/april.tex}
%\input{chapters/final.tex}

% \section{Ответы}


%\input{chapters/october_sol.tex}
%\input{chapters/december_sol.tex}
%\input{chapters/april_sol.tex}
%\input{chapters/final_sol.tex}



\section{2022-2023}

\subsection{2022-12-22 demo}


Rules: two parts, two hours in total, one A4 cheat sheet is allowed. 

\vspace{1cm}

Part 1. Test part, only numerical answers are checked, 6 questions, each question gives 1 point but no more than 4 points in total. 
This part is very predictable :) 

\begin{enumerate}
    \item (bootstrap) I have a sample $X_1$, \ldots, $X_{100}$.
    
    I generate one naive bootstrap sample $X^{*}_{1}$, \ldots, $X^{*}_{100}$. 

    What is the probability that the first observation will be present in the bootstrap sample 2 times or more?

    \item (welch) We have data for an $AB$-experiment $\bar X_a = 10$, $\bar X_b = 12$, 
    $n_a = 20$, $n_b = 30$, $\sum (X_i^a - \bar X_a)^2 = 100$, $\sum (X_i^b - \bar X_b)^2 = 200$.

    Calculate the standard error of $\bar X_a - \bar X_b$ for the Welch test. 
    
    \item (mw test) I have five results of two runners $A$ and $B$ for the 5 km race: 25:12 (A), 26:34 (B), 27:43 (A), 28:12 (A), 29:05 (B).
    
    Calculate Mann-Whitney statistic $U_A$ that tests the null-hypothesis of equal distributions of time. 
    
    (The statistic $U_A$ should positively depend on the ranks of the runner $A$). 

    \item (multiple comparison) I have 100 hypothesis with independent statistics. 
    The null hypothesis for all 100 cases is actually true, but I don't know this. 
    
    I calculate all p-values. 
    If the two lowest p-value are both lower than $0.05$ I wrongly conclude that not all $H_0$ are true. 
    Otherwise I correctly conclude that all $H_0$ are true. 

    What is the probability that I will get the correct conclusion?

    \item (sample size) My target variable is binary and I wish 
    minimal detectable effect equal to $0.01$, probability of I-error  not greater than $0.02$, 
    probability of II-error not greater than $0.10$, control and experimental group of the same size equal to $n$.

    What is minimal value of $n$?


    \item (anova 1+2) Vasiliy loves to eat shaurma. He has three local shaurma dealers. Vasiliy bought 7 shaurmas from each dealer. 
    and measured their weight. He would like to test the hypothesis that mean weight is the same for all dealers. 

    Total sum of squares is 1000, between sum of squares is 500. 

    Calculate the $F$-statistic to test the hypothesis.
        
\end{enumerate}

Part 2. Open part, solutions are required, 4 problems, each problem gives 2 points but no more than 6 points in total.
This part is almost unpredictable :)

\begin{enumerate}
 \item Let random variables $Y_1$, \ldots, $Y_n$ be iid uniform $U[0;1]$.
    Consider the naive bootstrap sample $Y_1^*$, \ldots, $Y_n^*$.

    Find $\Var(Y_1^*)$, $\Cov(Y_1^*, Y_2^*)$, $\Var(\bar Y^*)$.

    \item Winnie-the-Pooh simulteneously tests $h$ null hypothesis using independent samples. 
    All the null hypothesis are true but Winnie does not know it. 
    
    \begin{enumerate}
        \item What is the probability that the highest P-value will be greater than $0.95$?
        \item What is the possible range for the probability in point (a) if exactly one null hypothesis is false?
    \end{enumerate}

    \item The correlation matrix of standardized variables $a$, $b$ and $c$ is given by
    \[
    C = \begin{pmatrix}
        1 & 0.2 & 0 \\
         & 1 &  0.2 \\
         & & 1 \\
    \end{pmatrix}
    \]

    Let $p_1$, $p_2$ and $p_3$ be the principal components. 

    \begin{enumerate}
        \item Express $p_1$ in terms of $a$, $b$ and $c$. 
        \item Express $b$ in terms of $p_1$, $p_2$ and $p_3$. 
        \item How would you restore the second observation of variable $b$ if you know that first and second 
        components for the second observation are equal to $-1$ and $2$ respectively?
    \end{enumerate}

    \item Consider the Mann-Whitney test with possible ties. 
    The variables $X_1$, $X_2$, \ldots, $X_{n_x}$ are iid Poisson with rate $\lambda =1$. 
    The variables $Y_1$, $Y_2$, \ldots, $Y_{n_y}$ are iid Poisson with the same rate, independent from $X$ sample. 

    Let $L$ be the number of all pairs $(X_i, Y_j)$ such that $X_i > Y_j$.

    \begin{enumerate}
        \item Find $\E(L)$, $\Var(L)$. 
        \item What is the probability that the ordered sequence of all $X_i$ and $Y_j$ will start with 
        three or more members from $X$-sample?
    \end{enumerate}

\end{enumerate}

\subsection{2022-12-22 exam}

Part 1. Test part, only numerical answers are checked, 6 questions, each question gives 1 point but no more than 4 points in total. 

\begin{enumerate}


\item I have a sample \(X_1, \ldots, X_{90}\).
I generate one naive bootstrap sample \(X^*_1, \ldots, X^*_{90}\).
    
What is the probability that the first observation will be present in the bootstrap sample exactly 3 times?
\item  We have data of an AB experiment: \(\bar X_a = 5.4\), \(\bar X_b = 6\),
\(n_a = 18\), \(n_b = 15\), \(\sum (X_i^a - \bar X_a)^2 = 890\),
\(\sum (X_i^b - \bar X_b)^2 = 800\).

Calculate the estimate of variance of \(\bar X_a - \bar X_b\) for the Welch test.

\item I have five results of two runners A and B for the 5 km race:

16:49 (B), 21:17 (A), 18:30 (B), 6:18 (B), 20:16 (A), 15:39 (B).

Calculate Mann-Whitney statistic \(U_A\) that tests the null-hypothesis of equal distributions of time.

(The statistic \(U_A\) should positively depend on the ranks of the runner A).

\item  I have 30 hypothesis with independent statistics. The null hypothesis for all 30
cases is actually true, but I don't know this.

I calculate all p-values.
If the 4 lowest p-value are simultaneously lower than 0.01 I wrongly conclude that not all \(H_0\) are true. 
Otherwise I correctly conclude that all \(H_0\) are true.

What is the probability that I will get the correct conclusion?

\item My target variable is binary and I wish minimal detectable effect equal to 0.04, 
probability of I-error not greater than 0.01, probability of II-error not greater than $0.2$. 
The control and experimental group are of the same size equal to \(n\).

Which minimal value of \(n\) is sufficient in the worst case?

\item Vasiliy loves to eat shaurma. He has \(5\) local shaurma dealers. Vasiliy bought \(4\) shaurmas
from each dealer and measured their weight. He would like to test the hypothesis that mean weight is the
same for all dealers.

Total sum of squares is \(600\), between sum of squares is \(100\)
Calculate the F -statistic to test the hypothesis.

\end{enumerate}


Part 2. Open part, solutions are required, 4 problems, each problem gives 2 points but no more than 6 points in total.


\begin{enumerate}[resume]
    \item Let random variables $Y_1$, \ldots, $Y_n$ be iid uniform $U[0;1]$.
    Consider the naive bootstrap sample $Y_1^*$, \ldots, $Y_n^*$.
    
    Find $\P(Y_1^* = Y_1)$, $\Cov(Y_1^*, Y_2)$, $\P(\max\{Y_1, \ldots, Y_n\} = \max\{Y_1^*, \ldots, Y_n^*\})$.
    
    \item The eigenvalues of sample correlation matrix are $2.5$, $0.3$ and $0.2$. 
    The eigenvalue $\lambda=2.5$ corresponds to eigenvector $v=(3, 4, -2)$.

    \begin{enumerate}
        \item James Bond predicts every original variable using multivariate regression on the first two components.
        What is the average value of $R^2$ he will get? 
        \item Express the first principal component in terms of scaled original variables $a$, $b$ and $c$. 
    \end{enumerate}
    \item Winnie-the-Pooh simulteneously tests $h$ null hypothesis using independent samples. 
    All the null hypothesis are true but Winnie does not know it. 
    
    \begin{enumerate}
        \item What is the expected value of the lowest P-value?
        \item What is the expected number of wrongly regected hypothesis if Winnie rejects all the hypothesis with P-value less $0.1$?
    \end{enumerate}


\item There are three continuosly distributed samples of the same size $n$, $X_1$, \ldots, $X_n$, $Y_1$, \ldots, $Y_n$,
$Z_1$, \ldots, $Z_n$. Imagine that the null hypothesis that all samples have the same distribution is true. 

Consider the random variable $R_X$ — the sum of ranks of the $X$ sample in the pooled sample. 

\begin{enumerate}
    \item Find the expected value $\E(R_X)$.
    \item What is the probability that $R_X$ will be equal to $n(n+1)/2$?
\end{enumerate}


\end{enumerate}




\addcontentsline{toc}{section}{Answer, hints and solutions} % руками добавляем фейковую секцию Ответы в оглавление
\addtocontents{toc}{\protect\setcounter{tocdepth}{0}}% Allow only \chapter in ToC


\subsection{2022-12-22 demo solutions}


\subsection{2022-12-22 exam solutions}

\begin{enumerate}
    \item \begin{enumerate}
        \item (2 points) $\P(Y_1^* = Y_1) = 1/n$
        \item (4 points) 
        \[
        \Cov(Y_1^*, Y_2) = \Cov(I_1 Y_1 + \ldots + I_n Y_n, Y_2) = \Cov(I_2 Y_2, Y_2);    
        \] 
        \[
        \Cov(I_2 Y_2, Y_2) = \E(I_2 Y_2^2) - \E(I_2 Y_2)\E(Y_2) = \frac{1}{n}\Var(Y_2) = \frac{1}{12n}    
        \]
        \item (4 points)
        \[
        \P( \max\{Y_1^*, \ldots, Y_n^*\} = \max\{Y_1, \ldots, Y_n\}) = 1 - \left(\frac{n-1}{n}\right)^n \approx 1 - e^{-1}
        \]
    \end{enumerate}
    \item \begin{enumerate}
        \item (5 points) 
        \[
            \text{average }R^2 = \frac{2.5 + 0.3}{2.5 + 0.3 + 0.2} = \frac{28}{30} \approx 0.93
        \]
        \item (5 points)
        \[
        p = \frac{3}{\sqrt{29}}a + \frac{4}{\sqrt{29}}b - \frac{2}{\sqrt{29}}c    
        \]
    \end{enumerate}
    \item \begin{enumerate}
        \item (5 points) One possible solution: obtain the density of minimal p-value (3 points) and calculate expected value (2 poitns):
        \[
        f(m) = h (1 - m)^{h-1}, \text{ where } m \in [0;1]    
        \]
        \[
        \E(M) = \int_0^1 m f(m) dm = \frac{1}{h + 1}    
        \]
        \item (5 poinst) Note that $N_{\text{wrong}} \sim \mathrm{Bin}(h, 0.1)$. Hence $\E(N_{\text{wrong}}) = 0.1 h$.
    \end{enumerate}
    \item \begin{enumerate}
        \item (5 points)
        \[
        \E(R_X) = n \E(R(X_1)) = n \frac{1 + 3n}{2}    
        \]
        \item (5 points)
        \[
        \P(\text{X-obs are before other obs})  = \frac{1}{C_{3n}^n} = \frac{n!(2n)!}{(3n)!}   
        \]
    \end{enumerate}
\end{enumerate}
\end{document}
